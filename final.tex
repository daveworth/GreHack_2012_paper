%% This is a skeleton file demonstrating the use of IEEEtran.cls
%% (requires IEEEtran.cls version 1.7 or later) with an IEEE conference paper.
%%
%% Support sites:
%% http://www.michaelshell.org/tex/ieeetran/
%% http://www.ctan.org/tex-archive/macros/latex/contrib/IEEEtran/
%% and
%% http://www.ieee.org/

% *** Authors should verify (and, if needed, correct) their LaTeX system  ***
% *** with the testflow diagnostic prior to trusting their LaTeX platform ***
% *** with production work. IEEE's font choices can trigger bugs that do  ***
% *** not appear when using other class files.                            ***
% The testflow support page is at:
% http://www.michaelshell.org/tex/testflow/



% Note that the a4paper option is mainly intended so that authors in
% countries using A4 can easily print to A4 and see how their papers will
% look in print - the typesetting of the document will not typically be
% affected with changes in paper size (but the bottom and side margins will).
% Use the testflow package mentioned above to verify correct handling of
% both paper sizes by the user's LaTeX system.
%
% Also note that the "draftcls" or "draftclsnofoot", not "draft", option
% should be used if it is desired that the figures are to be displayed in
% draft mode.
%
\documentclass[conference]{IEEEtran}
% Add the compsoc option for Computer Society conferences.
% *** CITATION PACKAGES ***
%
\usepackage{cite}
% cite.sty was written by Donald Arseneau
% V1.6 and later of IEEEtran pre-defines the format of the cite.sty package
% \cite{} output to follow that of IEEE. Loading the cite package will
% result in citation numbers being automatically sorted and properly
% "compressed/ranged". e.g., [1], [9], [2], [7], [5], [6] without using
% cite.sty will become [1], [2], [5]--[7], [9] using cite.sty. cite.sty's
% \cite will automatically add leading space, if needed. Use cite.sty's
% noadjust option (cite.sty V3.8 and later) if you want to turn this off.
% cite.sty is already installed on most LaTeX systems. Be sure and use
% version 4.0 (2003-05-27) and later if using hyperref.sty. cite.sty does
% not currently provide for hyperlinked citations.
% The latest version can be obtained at:
% http://www.ctan.org/tex-archive/macros/latex/contrib/cite/
% The documentation is contained in the cite.sty file itself.

\usepackage{listings}

% *** GRAPHICS RELATED PACKAGES ***
%
\ifCLASSINFOpdf
  % \usepackage[pdftex]{graphicx}
  % declare the path(s) where your graphic files are
  % \graphicspath{{../pdf/}{../jpeg/}}
  % and their extensions so you won't have to specify these with
  % every instance of \includegraphics
  % \DeclareGraphicsExtensions{.pdf,.jpeg,.png}
\else
  % or other class option (dvipsone, dvipdf, if not using dvips). graphicx
  % will default to the driver specified in the system graphics.cfg if no
  % driver is specified.
  % \usepackage[dvips]{graphicx}
  % declare the path(s) where your graphic files are
  % \graphicspath{{../eps/}}
  % and their extensions so you won't have to specify these with
  % every instance of \includegraphics
  % \DeclareGraphicsExtensions{.eps}
\fi
% graphicx was written by David Carlisle and Sebastian Rahtz. It is
% required if you want graphics, photos, etc. graphicx.sty is already
% installed on most LaTeX systems. The latest version and documentation can
% be obtained at: 
% http://www.ctan.org/tex-archive/macros/latex/required/graphics/
% Another good source of documentation is "Using Imported Graphics in
% LaTeX2e" by Keith Reckdahl which can be found as epslatex.ps or
% epslatex.pdf at: http://www.ctan.org/tex-archive/info/

% *** MATH PACKAGES ***
%
%\usepackage[cmex10]{amsmath}
% A popular package from the American Mathematical Society that provides
% many useful and powerful commands for dealing with mathematics. If using
% it, be sure to load this package with the cmex10 option to ensure that
% only type 1 fonts will utilized at all point sizes. Without this option,
% it is possible that some math symbols, particularly those within
% footnotes, will be rendered in bitmap form which will result in a
% document that can not be IEEE Xplore compliant!
%
% Also, note that the amsmath package sets \interdisplaylinepenalty to 10000
% thus preventing page breaks from occurring within multiline equations. Use:
%\interdisplaylinepenalty=2500
% after loading amsmath to restore such page breaks as IEEEtran.cls normally
% does. amsmath.sty is already installed on most LaTeX systems. The latest
% version and documentation can be obtained at:
% http://www.ctan.org/tex-archive/macros/latex/required/amslatex/math/





% *** SPECIALIZED LIST PACKAGES ***
%
%\usepackage{algorithmic}
% algorithmic.sty was written by Peter Williams and Rogerio Brito.
% This package provides an algorithmic environment fo describing algorithms.
% You can use the algorithmic environment in-text or within a figure
% environment to provide for a floating algorithm. Do NOT use the algorithm
% floating environment provided by algorithm.sty (by the same authors) or
% algorithm2e.sty (by Christophe Fiorio) as IEEE does not use dedicated
% algorithm float types and packages that provide these will not provide
% correct IEEE style captions. The latest version and documentation of
% algorithmic.sty can be obtained at:
% http://www.ctan.org/tex-archive/macros/latex/contrib/algorithms/
% There is also a support site at:
% http://algorithms.berlios.de/index.html
% Also of interest may be the (relatively newer and more customizable)
% algorithmicx.sty package by Szasz Janos:
% http://www.ctan.org/tex-archive/macros/latex/contrib/algorithmicx/




% *** ALIGNMENT PACKAGES ***
%
%\usepackage{array}
% Frank Mittelbach's and David Carlisle's array.sty patches and improves
% the standard LaTeX2e array and tabular environments to provide better
% appearance and additional user controls. As the default LaTeX2e table
% generation code is lacking to the point of almost being broken with
% respect to the quality of the end results, all users are strongly
% advised to use an enhanced (at the very least that provided by array.sty)
% set of table tools. array.sty is already installed on most systems. The
% latest version and documentation can be obtained at:
% http://www.ctan.org/tex-archive/macros/latex/required/tools/


%\usepackage{mdwmath}
%\usepackage{mdwtab}
% Also highly recommended is Mark Wooding's extremely powerful MDW tools,
% especially mdwmath.sty and mdwtab.sty which are used to format equations
% and tables, respectively. The MDWtools set is already installed on most
% LaTeX systems. The lastest version and documentation is available at:
% http://www.ctan.org/tex-archive/macros/latex/contrib/mdwtools/


% IEEEtran contains the IEEEeqnarray family of commands that can be used to
% generate multiline equations as well as matrices, tables, etc., of high
% quality.


%\usepackage{eqparbox}
% Also of notable interest is Scott Pakin's eqparbox package for creating
% (automatically sized) equal width boxes - aka "natural width parboxes".
% Available at:
% http://www.ctan.org/tex-archive/macros/latex/contrib/eqparbox/





% *** SUBFIGURE PACKAGES ***
%\usepackage[tight,footnotesize]{subfigure}
% subfigure.sty was written by Steven Douglas Cochran. This package makes it
% easy to put subfigures in your figures. e.g., "Figure 1a and 1b". For IEEE
% work, it is a good idea to load it with the tight package option to reduce
% the amount of white space around the subfigures. subfigure.sty is already
% installed on most LaTeX systems. The latest version and documentation can
% be obtained at:
% http://www.ctan.org/tex-archive/obsolete/macros/latex/contrib/subfigure/
% subfigure.sty has been superceeded by subfig.sty.



%\usepackage[caption=false]{caption}
%\usepackage[font=footnotesize]{subfig}
% subfig.sty, also written by Steven Douglas Cochran, is the modern
% replacement for subfigure.sty. However, subfig.sty requires and
% automatically loads Axel Sommerfeldt's caption.sty which will override
% IEEEtran.cls handling of captions and this will result in nonIEEE style
% figure/table captions. To prevent this problem, be sure and preload
% caption.sty with its "caption=false" package option. This is will preserve
% IEEEtran.cls handing of captions. Version 1.3 (2005/06/28) and later 
% (recommended due to many improvements over 1.2) of subfig.sty supports
% the caption=false option directly:
%\usepackage[caption=false,font=footnotesize]{subfig}
%
% The latest version and documentation can be obtained at:
% http://www.ctan.org/tex-archive/macros/latex/contrib/subfig/
% The latest version and documentation of caption.sty can be obtained at:
% http://www.ctan.org/tex-archive/macros/latex/contrib/caption/




% *** FLOAT PACKAGES ***
%
%\usepackage{fixltx2e}
% fixltx2e, the successor to the earlier fix2col.sty, was written by
% Frank Mittelbach and David Carlisle. This package corrects a few problems
% in the LaTeX2e kernel, the most notable of which is that in current
% LaTeX2e releases, the ordering of single and double column floats is not
% guaranteed to be preserved. Thus, an unpatched LaTeX2e can allow a
% single column figure to be placed prior to an earlier double column
% figure. The latest version and documentation can be found at:
% http://www.ctan.org/tex-archive/macros/latex/base/



%\usepackage{stfloats}
% stfloats.sty was written by Sigitas Tolusis. This package gives LaTeX2e
% the ability to do double column floats at the bottom of the page as well
% as the top. (e.g., "\begin{figure*}[!b]" is not normally possible in
% LaTeX2e). It also provides a command:
%\fnbelowfloat
% to enable the placement of footnotes below bottom floats (the standard
% LaTeX2e kernel puts them above bottom floats). This is an invasive package
% which rewrites many portions of the LaTeX2e float routines. It may not work
% with other packages that modify the LaTeX2e float routines. The latest
% version and documentation can be obtained at:
% http://www.ctan.org/tex-archive/macros/latex/contrib/sttools/
% Documentation is contained in the stfloats.sty comments as well as in the
% presfull.pdf file. Do not use the stfloats baselinefloat ability as IEEE
% does not allow \baselineskip to stretch. Authors submitting work to the
% IEEE should note that IEEE rarely uses double column equations and
% that authors should try to avoid such use. Do not be tempted to use the
% cuted.sty or midfloat.sty packages (also by Sigitas Tolusis) as IEEE does
% not format its papers in such ways.





% *** PDF, URL AND HYPERLINK PACKAGES ***
%
%\usepackage{url}
\usepackage[pagebackref=true]{hyperref}
\hyphenation{op-tical net-works semi-conduc-tor}


\begin{document}
%
% paper title
% can use linebreaks \\ within to get better formatting as desired
\title{Leveraging Convention over Configuration for Static Analysis in Dynamic Languages}

\author{\IEEEauthorblockN{
  David Worth\IEEEauthorrefmark{1},
  Justin Collins\IEEEauthorrefmark{2}}

  \IEEEauthorblockA{\IEEEauthorrefmark{1}Highgroove Studios\\
  112 Krog St, Suite 6
  Atlanta, GA 30307\\
  Email: dave@highgroove.com
  Twitter: @highgroovedave}

  \IEEEauthorblockA{\IEEEauthorrefmark{2}Twitter\\
  Email: collins@twitter.com\\
  Twitter: @presidentbeef}
}

% make the title area
\maketitle


\begin{abstract}
%\boldmath
Static analysis in dynamic languages is a well known difficult problem in computer
science, with a great deal of emphasis being put on type
inference\cite{Furr:2009:STI:1529282.1529700}.  The problem is so
difficult that Holkner and Harland's paper on static analysis in
Python opens immediately with, ``The Python programming language is typical among dynamic languages in
that programs written in it are not susceptible to static
analysis.''\cite{Holkner:2009:EDB:1862659.1862665}  Dynamic languages such as Ruby provide
impressive programming power thanks to expressive language constructs and flexible typing.
Ruby, in particular, is strongly leveraged in the web development ecosystems thanks to
well known and supported frameworks such as Ruby on Rails and Sinatra.

Web application security is a particularly difficult area for a number of
reasons including, the low-barrier to entry for new developers combined with the
high-demand for their services, the increasing complexity of the web-based
ecosystem, and the traditional languages and frameworks for web-development not
adopting a strong defensive stance as their default.  Ruby on Rails adopts the
``convention over configuration'' policy aimed at aiding developers of all
levels in building robust web applications with a minimum of configuration.  The
goal is for the framework to simply ``do the right thing'' by default, and more
sophisticated features and technologies are to be explicitly applied by
developers with those more advanced requirements and understanding.  Much of the
power in the Ruby on Rails framework stems from careful use of ``magic''
functions: dynamically generated functions using Ruby's powerful metaprogramming
structures.  As a side effect, many of the methods called by developers are not
available to a static analysis tool by simply examining the code on disk.  We
are able to leverage the consistency of the language and framework to perform
static analysis on Ruby on Rails applications, and reason about their attack
surface.  This is done by analyzing the abstract syntax tree, and sometimes the
configuration (generally simply library versions) of the program
itself and by comparing it to a pre-compiled library of known security issues
exposed by the Ruby on Rails framework.
\end{abstract}

% For peer review papers, you can put extra information on the cover
% page as needed:
% \ifCLASSOPTIONpeerreview
% \begin{center} \bfseries EDICS Category: 3-BBND \end{center}
% \fi
%
% For peerreview papers, this IEEEtran command inserts a page break and
% creates the second title. It will be ignored for other modes.
\IEEEpeerreviewmaketitle

\section{Introduction}

Ruby on Rails is a popular web framework which provides a Model-View-Controller
architecture along with many ancillary tools to engineer complicated web
applications with a minimum of code as well as a minimum of exposed complexity.
The fundamental principal employed to reduce complexity is ``convention over
configuration'', meaning that standardized methods are used to achieve standard
functionality.  In many ways this philosophy resembles that of the Python
community and its ``There's only one way to do it'' philosophy.  The
advantage to such a philosophy is that one can successfully rely on the
conventions to expose large families of security vulnerabilities present in
modern web applications.  The Ruby on Rails Security Guide\cite{rails_security}
lists a comprehensive collection of general web application security
vulnerabilities, and a number of Rails-specific vulnerabilities, and their
mitigations as provided for by the framework.  Justin Collins, the original
author of the Brakeman Scanner for Ruby on Rails applications, exploited exactly
this convention-based approach, and these Rails-specific issues, in designing
the scanner.

\section{Brakeman Scanner Architecture}

At a high level Brakeman treats Ruby as an ``acceptable Lisp'' and uses
existing parsers to decompose the code into S-expressions\cite{sexp_for_rubyists}.  Each
s-expression can then be interrogated for its type, for example a function definition, a
collection of arguments, a Ruby block, a function call, or a string interpolation.  These
basic building blocks forming an abstract syntax tree are the fundamental objects used by
Brakeman, along with its knowledge of Rails conventions, to analyze a given Rails
application for potential security vulnerabilities.

Moreover, some basic ``taint flow analysis'' can be performed based on the conventions
within Rails.  The means by which user-input enters the system is fairly consistent, with
three of the major sources being the cookie collection, a parameters hash available in,
and used as the main source of input to, controller actions, and the request object which
wraps up the context of a given request to the application.  Due to the consistency of the
sources we may reason concisely about the danger presented by relying upon input from
those sources directly in contexts which have any potential security implications.

\subsection{Static Analysis of Ruby}

The fundamental idea behind Brakeman is that Rails is a framework, or a
domain specific language (DSL), for web development.  As such, tokens within Ruby and
Rails which appear to be analogous to ``keywords'' of other languages are
actually method calls in Ruby.  Rails is often attributed with ``magic''
functionality which to an experienced Rails developer is a side effect of a well
thought out, non-intrusive, DSL.

To convert a Ruby program, or Rails application, into S-Expressions for analysis
we use the combination of Racc\cite{racc} and RubyParser\cite{ruby_parser}.  Racc is a parser
generator, analogous to Yacc\cite{yacc} for converting grammars into parsers for the
grammar definition.  RubyParser provides two grammars, one each for the unique
syntactical structures of Ruby 1.8 and 1.9.  RubyParser then relies on the
parsers generated by Racc to provide generate S-Expressions upon which we can
operate.  The node-types in the S-Expressions produced by RubyParser are the
node types as defined by Ruby\cite{ruby_node_types} in its virtual machine.

It is interesting to note that the Ruby Parser and any given Ruby implementation
may actually differ in their interpretation or parsing of a given program, as
could any two implementations.  The Ruby ecosystem is fairly unique in that the
definitive reference for correctness of Ruby is Matz Ruby Implementation (MRI),
as the implementation moves very quickly and the definitive RubySpec has not
been fully translated out of Japanese.  An advantage of RubyParser being
implemented separately from a given implementation is that it is not susceptible
to any implementation bugs of a given implementation;  By the same token, it may
implement its own.

SexpProcessor\cite{sexp_processor} provides a very flexible, convention based
S-Expression processing framework.  An S-Expression Processor inherits from the
SexpProcessor class and simply defines methods of the form
$process\_\langle~ruby~node~type\rangle$ and
$rewrite\_\langle~ruby~node~type\rangle$.

Brakeman extends SexpProcessor's building in processor base-class to quickly
capture important, and potentially vulnerable, points of execution context.  It
does this in two phases: a pre-processing collection phase, during which potentially
vulnerable s-expressions are collected based on the type of vulnerability being
checked for, and an analysis phase where the pre-stored S-Expressions are
analyzed for actual vulnerabilities, and if one is found, it is rated in its
confidence.

For example, in pre-processing templates, for Cross-Site Scripting it examines
local variable assignments, and any point where there is an output from a Ruby
expression.  It is able to collect those expressions via a call to
$process\_output$, and store them for later analysis.  Later, in the analysis
phase, we retrieve the output points and check them for user controlled data. It
stores the results in a struct which contains the type of user controlled data
along with the S-Expression which represents the vulnerable code.

Finally these matches can be analyzed individually for the actual
vulnerabilities, and classify them appropriately in preparation for the
reporting phase.  This analysis phase requires walking the vulnerable
S-Expressions and interrogating them for use of particular variables, or copies
of particular variables.  For example, the data-structure which carries the
context of a given HTTP request in Rails is the \texttt{params} hash.  If this
hash is output directly without validation the program is vulnerable to a
cross-site scripting vulnerability, which is recorded and reported.  This is the
power of convention within a web framework such as Rails.  If there were no
convention, but rather programmer defined naming schemes for request parameters,
a great deal more work would be required in terms of data-flow analysis and
taint-checking to determine if a program is susceptible to cross-site scripting.
By leaning upon best practices and convention we can easily reason about the
safety, or lack there-of, of various expressions.

\subsection{The General Case}

Given a language with a well-defined grammar and a parser, we can construct an
abstract syntax tree (AST) for a program written in that language.  Examining
that AST will allow us to reason about the properties of the given program.  The
strength of Brakeman comes less from the nature of Ruby and its abstract syntax
tree, or its representation as S-Expressions, but rather from leveraging
convention over configuration.  One such convention, mentioned above, is that
all HTTP request parameters are exposed in a consistently named variable:
\texttt{params} within a controller action handling the request.  In this case,
there is nothing unique about Ruby or Rails in our ability to reason about the
contents of \texttt{params}.

The Python language includes in its Standard Library the necessary tools for
parsing Python and interrogating its abstract syntax tree via the ``Python
Language Services''\cite{python_language_services} tools.  The Grok\cite{grok}
project touts itself as a convention over configuration web framework with much
of the same functionality as Rails.  With this pair of powerful tools similar
static analysis should be possible to that which we perform in the Brakeman
scanner.  In examining the constructs provided by Grok there are some
differences which make such analysis more difficult than in the Ruby/Rails case.
For example, HTTP request parameters are named by the developer rather than by
the framework, so detecting XSS, for example, does require some variety of
data-flow analysis and taint-checking.  This short-coming is not a short-coming
of Grok as a powerful framework, but rather a failing of its \emph{convention}
based design.  Such a framework could be driven towards a more convention based
design to enhance its ability to be automatically analyzed and reasoned about.

The Groovy\cite{groovy} programming language is a dynamic language built on the
Java Virtual Machine built with the goals of implementing Ruby style DSLs and
other dynamic language techniques with the advantages of static types and
compilation.  The Grails\cite{grails} web framework is a convention over
configuration based web framework meant to very closely resemble the Rails
framework.  Its resemblance to Rails is so strong in fact that many of the
language constructs appear to be copied directly from Rails into Groovy with the
most minimal syntax changes possible.  As such, HTTP request
parameters are passed to controller actions as \texttt{params}, and thus their
use can be reasoned about in a similar fashion as we do in Brakeman for Rails
applications.  The CodeNarc\cite{codenarc} project provides some static analysis
tools for Groovy, though it is aimed at best-practices, and other Groovy
specific issues.  It would be an interesting exercise for the Groovy/Grails
community to port some of the functionality from Brakeman to CodeNarc in an
effort to enhance the security of Grails applications.

While the above comparisons focus on a single, fairly trivial use of convention
over configuration, specifically the use of a single name for the HTTP
parameters passed to a given controller action, it is illustrative.  If all
system inputs are provided in easily recognized, and automatically constructed
fashions, then the onus of identification of potentially dangerous data within
an application is taken from the developer or their tools.  The focus of tools
can then be placed on other functionality such as detecting dangerous uses of
said data.

\section{Insecure User Input}

The standard adage of ``never trust user(-controlled) input'' is easily
proffered but one of the fundamental realities of developing applications
is that we \emph{must} act upon their data in some way.  As such it is key to
limit the ways in which we trust user-controlled data rather than simply
attempting to avoid the problem.

\subsection{SQLi Detection}

SQL Injection (SQLi), along with cross-site scripting (XSS) and cross-site request forgery
(CSRF), represent the most well-known, common, and language independent web vulnerability
classes.  Ruby on Rails, like any other web framework, can be made susceptible to such
flaws, though it does provide a number of mitigations and best practices to prevent them.
Moreover, many of the methods which automatically generate SQL are not explicitly defined
by either Ruby on Rails itself or by the user, but rather are automatically generated at
runtime by Rails.  The generation of these methods is based on interrogation of the
database which backs a given model wherein each column in the database is mapped to an
attribute of the model via the Object Relational Model (ORM).  Thus for a given user model
with two database columns ``name'' and ``email'', the User class itself responds to a
variety of methods, some explicitly defined such as \texttt{User.all}, which returns all
users in the system, and dynamic methods based on its attributes such as
\texttt{User.find\_by\_name}, \texttt{User.find\_by\_email},
\texttt{User.find\_by\_name\_and\_email}, and even
\texttt{User.find\_all\_by\_name\_and\_email}.  While using these finders generates
appropriate SQL automatically, and the parameters themselves are correctly isolated, this
is a perfect example of Rails conventions that are so consistent they can be trivially
identified in a static analysis context.

A naive approach one might take is to identify these finders is with a simple regular
expression.  Brakeman opts to use the abstract syntax tree from a Ruby parser as it
provides a deeper view into the application, but the above would work quickly for a given
model.  This consistent, convention-based, dynamic function generation is easily and
deterministically identified allowing for further reasoning about the functionality of the
code.

Compounding slightly the complexity of isolating SQLi is that the parameters to various
finders may very well contain user-controlled data, in this case passed in via a
parameters hash(-table).  Thus a simple, incorrect and explicitly recommended against
\cite{rails_security}, idiom for dynamically
finding a user by name would be \texttt{User.where(``name = \#\{params[user\_name]\}''}.
During the process of analysis we are able to reason simply about the
potential safety of such an expression.  The call signature of the
\texttt{where} method on an active record model only has a few different
permutations.  In the case where the first variable is a string it is to be
interpreted as ``raw'' SQL.  By examining the s-expression of the first
parameter we can quickly determine if \emph{any} (potentially user-controller)
variable is interpolated, and if so it is flagged as potentially dangerous.  In
the case where the user cannot control the data the flagged expression may not
be exploitable but is still an indication of an expression that should be
refactored to follow the best-practice of passing parameters to the ORM outside
of the query logic\cite{rails_security_sqli}.

If a programmer attempts a ``clever'' independent re-implementation of such
dynamic finders as described above, it is fundamentally more difficult to reason
about the safety of a given expression.  This is because there are no well understood and
accepted conventions about call signatures and handling as have been created during the
development of the Ruby on Rails framework.  Those conventions alone allow us to reason
concretely about the safety of a given expression.

\subsection{Cross-Site Scripting}

Within a given Rails application there are two sources of potential Cross-Site
Scripting vulnerabilities: Rails itself and user code.  Brakeman is
Rails-version aware, and adds checks for vulnerabilities within the Rails
framework as they are made public.  When such vulnerabilities are created with a
CVE, as they generally are, the warnings produced reference the given CVE and
the minimum version to which a developer must upgrade in order to no longer be
vulnerable to the given bug.

The second phase of detecting XSS, briefly outlined above, involves parsing the
templates within the system and examining them for outputs.  Each templating
language may have its own mechanism for denoting outputs but two of the most
common, Erb, and Haml, denote output as following a single \texttt{=} token.  By
identifying all of the points where output is produced from a Ruby expression we
can focus on identifying potentially dangerous inputs.  This is the
pre-processing phase discussed in general above.  Each new templating language
introduced to the Rails ecosystem must have a custom parser implemented to
support such scanning since their semantics may vary radically.

The bulk of the work in the analysis stage is performed by the
\texttt{has\_immediate\_user\_input?} and \texttt{has\_immediate\_model?}
predicates, which are called on all un-escaped output nodes detected by the
scanner.  The first of these predicates recursively walks the entire expressions
looking for access to the \texttt{params}, \texttt{cookies}, or \texttt{request}
hashes, as well as string interpolations and other potential sources of
user-controlled data.  Upon finding them a Match struct is returned with the
vulnerable expression.  The second predicate simply looks for calls directly on
a model class.  For example, if \texttt{Model.find\_by\_dangerous\_attribute} is
called directly with user-controlled input,  then a user may circumvent security
mechanisms or cause the system to leak secure data.  This may be converted to a
XSS if the returned data can be forced by the user to refer to a given model
stored in the database which they control.  For example, if the user is able to
store dangerous Javascript in their profile in a social-networking site, then
with a carefully crafted URL they might force another user to view their profile
which executes that Javascript.

Note that in the case of Rails 3.x, output in views is automatically escaped so
the checks become relatively trivial as the built-in escaping method is called
automatically.  With that in mind, there have been XSS vulnerabilities within
Rails itself, and the new mitigations should not be viewed as a claim that Rails
is somehow immune to this classic vulnerability.

\subsection{Insecure Redirect and Rendering}

A common idiom for concluding a controller action in a Ruby on Rails application
is to set some sort of notification to be displayed to the user, then a
destination is specified.  The destination to which the user is redirected is
specified by an options hash, and must not be user-controlled.  If the user does control the
parameters to the redirect call then they may manipulate the \texttt{host} attribute,
which can be used to send a user, after completing the initial action, to a remote
malicious host\cite{OWASP_redirect}.  The user will likely trust a given link to a trusted
site without realizing the host manipulation is occurring. To increase the confidence of
the victim, the attacker may obfuscate the vulnerability by making the site to which they
redirect them look identical to the trusted site if they control the destination.
Identification of such manipulation at runtime is simply done by identifying any variable
parameters to the redirect.  In that case the redirect is flagged as vulnerable.  The
Rails framework provides mitigations for such redirection via the \texttt{only\_path}
option, which can be easily applied by a developer upon being alerted to the
vulnerability.

Similarly, allowing a user to control the view or template to be rendered in a given
action may result in an information leak about the system by effectively bypassing any
authorization controls in place.  Rendering within Rails is done via the
\texttt{render}\cite{render_method} method, which expects to receive the name of a partial
as a string and an options hash.  The path to the rendered view template should never
contain user-controlled parameters to avoid this vulnerability.  Validating that it does
not simply means checking for string interpolation in the first parameter, and checking
for its source.  If the source is the cookie collection, the parameters hash as created by
Rails, or the request object then the source is deemed to be tainted and is flagged.  Much
like the taint-safety of Perl, once a variable from one of those sources has been
manipulated by the developer it is deemed untainted and can no longer be reasoned about.

\section{Explicit Access Controls via Exposed APIs and DSLs}

Much of the Rails architecture revolves around explicit APIs, exposed to the user to
perform various actions, along with domain-specific languages.  Domain-specific languages
(DSLs) are Ruby code which exploit the syntactic flexibility of Ruby itself to resemble a
unique language intended to express programmatic needs of the developer.

\subsection{Mass Assignment}

After the well-publicized compromise of Github by Egor Homakov\cite{egor}, the subject of
Mass Assignment vulnerabilities within the Rails framework came, once again, to the fore.
This vulnerability had been well understood within the community, and as such standardized
mitigations had been created and well documented.  The mitigations themselves were active
mitigations requiring understanding of the domain models to implement.  The goal was to
make such mitigations simple enough they could be implemented quite quickly with
fundamental understanding of the models and the use of their attributes.  With the Rails
3.2.6 release, mass-assignment has been disabled in development in such a way that an exception
is thrown if mass-assignment restrictions are violated at runtime as an aid to developers.
The vulnerability is triggered via the very convenient \texttt{update\_attributes} method
which is provided to simultaneously update multiple attributes on a model.  This may
optionally include so-called ``nested-attributes'', those attributes belonging to domain
models associated with the model being updated explicitly, via a single method call.  This
update is done without regard to the nature of those attributes, as Rails does not attempt
to reason about the use of any attribute.  For example, a user may have a boolean
attribute to indicate the user is an administrator.  Other attributes may be explicitly
exposed to the user through a web-form such as ``real name'' and ``home town''. In this
case the user themself might manipulate that web-form's fields to include an administrator
attribute of their own, as if it were provided by the developer, effectively making
themselves an administrator.

While current static analysis tools cannot prevent the above scenario, as they
cannot reason about the relative sensitivity of various fields within the system,
they can reveal those models which lack explicit white- or black-listing of attributes for
update.  Rails provides two mechanisms for attribute access control: white-listing via
\texttt{attr\_accessible}, and black-listing, the considerably less preferable route, via
\texttt{attr\_protected}.  The latter mechanism is often used in ``kitchen-sink'' models
which have so many attributes such that explicitly white-listing those attributes which should
be writeable is tedious for the developer, so a black-listing of the few sensitive
attributes is provided.  The \texttt{attr\_accessible} and \texttt{attr\_protected}
``keywords'' are actually methods in a sophisticated DSL.  We can identify calls to these
methods, and from their existence, infer that the developers of a given application have
at least taken steps to limit access to model attributes, and hopefully, reasoned about
their relative sensitivity. Though this is not a foolproof methodology,  it does indicate
to a developer or external auditor those models which have not explicitly made public
declarations about the sensitivity of their underlying attributes.
Moreover, those models which are of sufficient complexity to warrant a black-listing via
\texttt{attr\_protected} may warrant a significant refactor into smaller models, each of
which can have explicit access controls applied to them.  In this case the mere existence
of the black-list indicates a ``pain-point'' ripe for a refactor to improve the
architecture and security of the application in one pass.

\section{External library conventions}

Not only does the Rails framework itself encourage standardization in the
methods by which various pieces of functionality are achieved, but the entire
ecosystem has evolved with the same philosophy in mind.  Some libraries and
executable gems such as Bundler and Rake, have become so pervasive as to be considered an
integral part of developing a modern Rails application.

\subsection{Bundler}

With the introduction of the Bundler gem the process of creating, updating, and
maintaining library dependencies for Rails applications, a process which had
previously been notoriously difficult, has become trivial.  Bundler interrogates
a \texttt{Gemfile} and determines which versions of a given gem and each of its
dependencies (recursively) are required.  For example, an application may require
a gem by name, by name and explicit version (useful for specifying a bug-fix
level of a gem upon which an application relies), or an optimistic version which will never be less than the
specified version, but may choose minor versions up to the next major version
of a gem.  These options allow the system to avoid API changes that may be introduced into a well
semantically-versioned gem while staying abreast of bugfixes.  The results of the
dependency resolution process are stored in a \texttt{Gemfile.lock} which can be
interrogated for specific vulnerabilities.  This interrogation process does rely on some
sort of external library of vulnerabilities but does have some strong advantages.  The
Rails framework itself is installed as a gem from the \texttt{Gemfile}.  As
vulnerabilities are found within Rails itself, CVEs are created, and new versions are
released, the Brakeman scanner can be updated to check that a given application is not
vulnerable to those bugs fixed in various patch releases.  This is not formal ``static
analysis'' in the context of analyzing executable code but does follow the same philosophy
of simply relying on developers and tools to act consistently and openly such that the
code produced may be checked for correctness with a maximum of ease.

\subsection{Extension of Rails by Domain-Specific Languages}

Much of Rails which appears at first glance to be made of ``keywords'', as a
developer might be experienced with from other languages, is actually a function
call within the current class (or one of its ancestors).  Examples of this are the
\texttt{private}\cite{private_method} ``keyword'' for marking methods as inaccessible to
call from outside of the current class or module, and \texttt{attr\_accessible} and
\texttt{attr\_protected} for limiting access to a models attributes.  Thus, writing
specific handlers to analyze the parameters to these apparent keywords is no different
than writing handlers for an other function call.  Moreover, this means that extending
Brakeman to support external gems which introduce DSL methods of their own is of the same
difficulty as implementing handlers for the attribute control methods.  For example, the
MetaWhere gem by Ernie Miller which exposes some attribute-level searching of
ActiveRecord models, provides the \texttt{assoc\_searchable} and
\texttt{assoc\_unsearchable} methods to white and blacklist attribute in much the same
fashion as Rails attribute access methods.

\section{Conclusion}

By leveraging conventions within modern languages and frameworks, one can more easily write
sophisticated static analysis tools than one might in a \emph{laissez-faire} language which
encourages each developer to implement fundamental operations on their own.  Following the
conventions, we are able to reason strongly about a developer's use of given methods and
constructs with a minimum of overhead, and often without resorting to extremely low-level
analysis.  Ruby's type-system, while being intentionally flexible, lends itself to analysis
of its abstract syntax tree, and thus to determining when various pieces of data may be
user-controlled and when not.  Just as the Rails framework continues to evolve, so does
the Brakeman scanner with constant improvements in identifying security vulnerabilities in
an extremely popular web framework.

\section*{Acknowledgments}

The authors would like to thank the entire Brakeman Scanner team for their support in
improving the scanner constantly.  Moreover, David Worth would like to thank the project's
maintainer and original implementor Justin Collins for his encouragement to contribute to
and participate in the project via contributions to, and papers such as this.


% trigger a \newpage just before the given reference
% number - used to balance the columns on the last page
% adjust value as needed - may need to be readjusted if
% the document is modified later
%\IEEEtriggeratref{8}
% The "triggered" command can be changed if desired:
%\IEEEtriggercmd{\enlargethispage{-5in}}

% references section

% can use a bibliography generated by BibTeX as a .bbl file
% BibTeX documentation can be easily obtained at:
% http://www.ctan.org/tex-archive/biblio/bibtex/contrib/doc/
% The IEEEtran BibTeX style support page is at:
% http://www.michaelshell.org/tex/ieeetran/bibtex/
%\bibliographystyle{IEEEtran}
% argument is your BibTeX string definitions and bibliography database(s)
%\bibliography{IEEEabrv,../bib/paper}
%
% <OR> manually copy in the resultant .bbl file
% set second argument of \begin to the number of references
% (used to reserve space for the reference number labels box)
%\bibliographystyle{IEEEtran}
%\bibliography{IEEEabrv,CFP_response}

\begin{thebibliography}{1}
    \providecommand{\url}[1]{#1}
    \csname url@samestyle\endcsname
    \providecommand{\newblock}{\relax}
    \providecommand{\bibinfo}[2]{#2}
    \providecommand{\BIBentrySTDinterwordspacing}{\spaceskip=0pt\relax}
    \providecommand{\BIBentryALTinterwordstretchfactor}{4}
    \providecommand{\BIBentryALTinterwordspacing}{\spaceskip=\fontdimen2\font plus
      \BIBentryALTinterwordstretchfactor\fontdimen3\font minus
    \fontdimen4\font\relax}
    \providecommand{\BIBforeignlanguage}[2]{{%
      \expandafter\ifx\csname l@#1\endcsname\relax
      \typeout{** WARNING: IEEEtran.bst: No hyphenation pattern has been}%
      \typeout{** loaded for the language `#1'. Using the pattern for}%
      \typeout{** the default language instead.}%
      \else
      \language=\csname l@#1\endcsname
      \fi
    #2}}
    \providecommand{\BIBdecl}{\relax}
    \BIBdecl

  \bibitem{Furr:2009:STI:1529282.1529700}
    \BIBentryALTinterwordspacing
    M.~Furr, J.-h.~D. An, J.~S. Foster, and M.~Hicks, ``Static type inference for
    ruby,'' in \emph{Proceedings of the 2009 ACM symposium on Applied Computing},
    ser. SAC '09.\hskip 1em plus 0.5em minus 0.4em\relax New York, NY, USA: ACM,
    2009, pp. 1859--1866. [Online]. Available:
    \url{http://doi.acm.org/10.1145/1529282.1529700}
    \BIBentrySTDinterwordspacing

  \bibitem{Holkner:2009:EDB:1862659.1862665}
    \BIBentryALTinterwordspacing
    A.~Holkner and J.~Harland, ``Evaluating the dynamic behaviour of python
    applications,'' in \emph{Proceedings of the Thirty-Second Australasian
    Conference on Computer Science - Volume 91}, ser. ACSC '09.\hskip 1em plus
    0.5em minus 0.4em\relax Darlinghurst, Australia, Australia: Australian
    Computer Society, Inc., 2009, pp. 19--28. [Online]. Available:
    \url{http://dl.acm.org/citation.cfm?id=1862659.1862665}
    \BIBentrySTDinterwordspacing

  \bibitem{citing_static_type_inference_for_ruby}
    \href{http://scholar.google.com/scholar?cites=12614894093793389341&as_sdt=40000005&sciodt=0,22&hl=en}{Google Scholar results for ``Static Type Inference for Ruby''}

  \bibitem{brakeman}
    Justin Collins. \emph{BrakemanScanner} [Online]. Available: \url{http://brakemanscanner.org} (URL)

  \bibitem{rails_security}
    \emph{Ruby on Rails Security Guide} [Online]. \url{http://guides.rubyonrails.com/security.html} (URL)

  \bibitem{rails_security_sqli}
    \emph{Ruby on Rails Security Guide} [Online]. \url{http://guides.rubyonrails.org/security.html\#sql-injection} (URL)

  \bibitem{sexp_for_rubyists}
    Magnus Holm. (2010, February 4). \emph{Sexp for Rubyists} [Online]. \url{http://blog.rubybestpractices.com/posts/judofyr/sexp-for-rubyists.html} (URL)

  \bibitem{OWASP_redirect}
    \emph{OWASP Top 10 - Redirect} [Online]. \url{https://www.owasp.org/index.php/Top_10_2010-A10} (URL)

  \bibitem{egor}
    Egor Homakov (2012, March 4) \emph{wow how come I commit in master? O\_o} [Online]. \url{https://github.com/rails/rails/commit/b83965785db1eec019edf1fc272b1aa393e6dc57} (URL)

  \bibitem{render_method}
    \emph{Rails API Documentation} - \texttt{ActionView::Template\#render} [Online]. \url{http://apidock.com/rails/ActionView/Template/render} (URL)

  \bibitem{private_method}
    \emph{Rails API Documentation - \texttt{Module::private}} [Online]. \url{http://apidock.com/ruby/Module/private} (URL)

  \bibitem{racc}
    Aaron Patterson. \emph{Racc} [Online].  Available: \url{https://github.com/tenderlove/racc} (URL)

  \bibitem{ruby_parser}
    SeattleRB (Ryan Davis). \emph{RubyParser} [Online].  Available: \url{https://github.com/seattlerb/ruby_parser} (URL)

  \bibitem{sexp_processor}
    SeattleRB (Ryan Davis). \emph{SexpProcessor} [Online].  Available: \url{https://github.com/seattlerb/sexp_processor} (URL)

  \bibitem{yacc}
    Stephen C Johnson. \emph{Yacc} [Online].  Available: \url{http://dinosaur.compilertools.net/yacc/index.html} (URL)

  \bibitem{ruby_node_types}
    Ruby core team. \emph{Ruby source code} [Online].  Available: \url{https://github.com/ruby/ruby/blob/trunk/node.c\#L100-886} (URL)

  \bibitem{grok}
    Grok core team. \emph{Grok} [Online].  Available: \url{http://grok.zope.org/} (URL)

  \bibitem{python_language_services}
    Python core team.  \emph{Python Language Services} [Online].  Available: \url{http://docs.python.org/library/language.html} (URL)

  \bibitem{groovy}
    Groovy core team. \emph{Groovy} [Online].  Available: \url{http://groovy.codehaus.org/} (URL)

  \bibitem{grails}
    Grails core team. \emph{Grails} [Online].  Available: \url{http://grails.org/} (URL)

  \bibitem{codenarc}
    CodeNarc team.  \emph{CodeNarc} [Online].  Available: \url{http://codenarc.sourceforge.net/} (URL)

\end{thebibliography}

\end{document}


